%! Author = gince
%! Date = 1/20/2024

%-----------------------------------------
% Frame du document
%-----------------------------------------
\usepackage[left=2cm,right=2cm,top=2cm,bottom=2cm]{geometry}  % Définit les dimensions des marges
\usepackage[french]{babel}
\usepackage[utf8]{inputenc}
\usepackage[T1]{fontenc}
\usepackage{helvet}

%-----------------------------------------
% Core
%-----------------------------------------
\usepackage{fancyhdr}           % Numérotation des pages, headers
\fancyhf{}
\usepackage{hyperref}           % Références hypertexte
\usepackage{booktabs,multirow,hhline}        % Tableaux
\usepackage{graphicx}           % Figures
\usepackage{subfig,wrapfig,caption}        % Sous-figures, ancrées, mise en forme des captions
\usepackage{titlesec}            % Mise en forme des sections
\usepackage{enumitem}           % Personnaliser les énumérations
\usepackage{color}            % Texte de couleur
\usepackage[dvipsnames]{xcolor}         % Plus de couleurs funky
\usepackage{textcomp}
\usepackage{lastpage}
\usepackage{csquotes}
\usepackage{anyfontsize}

\pagestyle{fancy}            % Formattage par défaut des paragraphes
\parindent=0pt
\parskip=6pt
\setlength{\headheight}{15pt}
\setlength {\marginparwidth }{2cm} 

%-----------------------------------------
% Math
%-----------------------------------------
\usepackage{amsmath,amssymb,amsthm,nicefrac}      % Symboles et versatilité mathématique
\usepackage{mathrsfs}           % ¯\_(ツ)_/¯ Polices d'écriture en math
\usepackage{wasysym,marvosym}         % Autres symboles math
\usepackage{mathtools}           % Peaufine la configuration des équations (when used)
\usepackage{dsfont}
\usepackage{algorithm}
\usepackage{algpseudocode}

%-----------------------------------------
% Physique
%-----------------------------------------
\usepackage{tikz}            % Dessine des figures
%\usepackage[american]{circuitikz}         % Dessine des schémas de circuits électroniques
\usetikzlibrary{quantikz}

\usepackage{verbatim}           % Je l'utilise pour écrire en verbatim et pour les commentaires
%\usepackage{minted}           % Ajoute du langage de prog élégamment
\usepackage{lipsum}            % Génère le lorem ispum
\usepackage{siunitx}            % Unités du système international avec \si
\usepackage{cancel}
\usepackage{todonotes}
\usepackage{empheq}
\usepackage{physics}
\usepackage{bm}% Your new best friend in LaTeX
% http://ctan.math.ca/tex-archive/macros/latex/contrib/physics/physics.pdf      La documentation dudit merveilleux package

\usepackage{float}

\usepackage{halloweenmath} % décorations d'halloween pour vos devoirs (J'encourage l'utilisation de la sorcière mathémagique avec \mathwitch )

\def\CQFD{\begin{flushright}CQFD.\end{flushright}}
\def\RANCHITUP{\begin{flushright}CQFD.\end{flushright}}
\def\PIFPAF{\begin{flushright}CQFD.\end{flushright}}
% \RANCHITUP ou \PIFPAF pour écrire un CQFD bien placé

\newcommand{\uvec}[1]{\boldsymbol{\hat{\textbf{#1}}}}
\newcommand{\uveci}{{\bm{\hat{\textnormal{\bfseries\i}}}}}
\newcommand{\uvecj}{{\bm{\hat{\textnormal{\bfseries\j}}}}}
% Beaux vecteurs unitaires avec \uvec


\newcommand{\del}[2]{\frac{\partial #1}{\partial #2}}
\newcommand{\delp}[1]{\frac{\partial }{\partial #1}}
\newcommand{\ddfrac}[2]{\frac{\dd #1}{\dd #2}}
\newcommand{\ddfracp}[1]{\frac{\dd }{\dd #1}}
\newcommand{\braAket}[3]{\left<#1\left|#2\right|#3\right>}
\newcommand{\kbproj}[1]{\ketbra{#1}{#1}}

%-----------------------------------------
% Références
%-----------------------------------------
\numberwithin{table}{section}
\numberwithin{figure}{section}
\numberwithin{equation}{section}


%-----------------------------------------
% Entête
%-----------------------------------------
% Entête gauche-droite
\fancyhead[R]{\chaptermark}
\fancyheadoffset{0 cm}


\newcommand{\signature}[2]{
\noindent\begin{minipage}{0.5\textwidth}\raggedright
    #1
\end{minipage}
\hfill
\begin{minipage}{0.5\textwidth}\raggedleft
    \textbf{Date}: \underline{\hspace{#2}} \\
    \textbf{Signature}: \underline{\hspace{#2}}
\end{minipage}
}